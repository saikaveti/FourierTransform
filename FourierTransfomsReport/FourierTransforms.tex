\documentclass{amsproc}
\usepackage{indentfirst}
\usepackage{amsthm}
\usepackage{amsmath}
\usepackage{amsfonts}
\usepackage{amssymb}
\usepackage{hyperref}
\usepackage{url}


\title{The Fourier Transform: Variations and Applications}

\author{Sailesh Kaveti}

\begin{document}

\maketitle

\section{The Fourier Transform}

\subsection{Developing an Intuition for the Fourier Transform}

\mbox{}	\\
\indent Imagine we are given we are given a bucket of paint, and asked to replicate the color of that bucket. At first, it seems easy to simply describe it nominally, calling the bucket of paint as red, yellow, blue, or any other color.  This is a valid first instinct. However, imagine you were given a purple bucket of paint. Even in this instance, most people are familiar with the color purple and we could just mix equal parts of blue and red. Imagine you were given a bucket of paint that was dark purple. We could still call this bucket “dark purple”, but we would immediately run into problems when replicating the paint color. Do we need to put the colors on a 2:1 ratio or a 3:1 ratio? In general, given a color, it can be very useful to know the colors that compose that color, as opposed to just a nominal description. Now imagine that we were given a complex wave, and that we are interested in the simpler waves that compose that wave. This is the value of the Fourier Transform.

\subsection{The Fourier Series}

\mbox{}	\\
\indent In order to understand the Fourier Transform, it is very important to gain understand the Fourier Series and its integral and the Fourier Transform is non-discretized extension of the Fourier Series.

\subsection{The Computation of the Fourier Transform}

\section{The Discrete Fourier Transform}

\subsection{The Basics of the Discrete Fourier Transform}

\subsection{How the Discrete Fourier Transform varies from the Fourier Transform}

\section{The Computation of the Discrete Fourier Transform}

\subsection{A Rudimentary Algorithm for the Discrete Fourier Transform}

\subsection{The Fast Fourier Tranform}

\section{Applications of the Fourier Transform}

\subsection{Convolution of Polynomials}

\subsection{Image Processing}

\end{document}